\documentclass[12pt,a4paper]{article}
\usepackage[utf8]{inputenc}
\usepackage{geometry}
\geometry{a4paper, margin=1in}
\usepackage{titlesec}
\usepackage{hyperref}
\usepackage{xcolor}
\usepackage{ragged2e} % For justification

% Hyperlink settings for blue-colored links
\hypersetup{
    colorlinks=true,
    linkcolor=blue,
    filecolor=blue,
    citecolor=blue,
    urlcolor=blue
}

% Title formatting
\titleformat{\section}{\large\bfseries}{\thesection.}{1em}{}
\titleformat{\subsection}{\normalsize\bfseries}{\thesubsection.}{1em}{}

% Title Information
\title{\textbf{Machine Learning in Aviation: Challenges and Opportunities}}
\author{Shalini Sur \\ Roll No: 22052154 \\ KIIT University Bhubaneshwar}
\date{}

\begin{document}

\maketitle

\justify
Machine Learning (ML) has emerged as a transformative force in the aviation industry, enabling advancements in predictive maintenance, air traffic management, and passenger experience. However, implementing ML models in a regulated, safety-critical environment presents unique challenges. The aviation industry is adopting ML technologies to improve safety, operational efficiency, and passenger satisfaction. Predictive maintenance, delay prediction, and air traffic optimization are notable applications. Despite these advancements, deployment of ML in this field is constrained by strict regulations, the need for high accuracy, and dynamic environmental factors.

During this research, challenges included limited access to aviation data due to confidentiality, ensuring the reliability of ML models in critical applications, and addressing model interpretability for compliance with regulatory standards. Similarly, challenges identified in published research include integrating data from diverse sources like aircraft sensors and weather systems, meeting high computational demands for real-time applications, addressing privacy concerns related to sensitive data, and managing scalability for growing data volumes. To overcome these issues, fostering collaboration between industry and academia to improve data sharing, leveraging cloud computing for scalability and processing needs, adopting ethical guidelines for privacy and data security, and utilizing advanced simulation environments for testing are recommended.

\section*{Keywords}
Machine Learning, Aviation, Predictive Maintenance, Air Traffic Management, Safety

\section*{References}
\begin{enumerate}
    \item Author 1. (2024). \textit{Aircraft Predictive Maintenance: An Application of Machine Learning Algorithms}. Available at: \href{https://www.researchgate.net/publication/381924182_AIRCRAFT_PREDICTIVE_MAINTENANCE_AN_APPLICATION_OF_MACHINE_LEARNING_ALGORITHMS_VERSHA_INTERIM_REPORT_JULY_2024_2}{ResearchGate}.
    \item Author 2. (2024). \textit{Real-Time Air Traffic Optimization Using Machine Learning}. Available at: \href{https://rdcu.be/d6ctv}{Springer}.
    \item Author 3. (2024). \textit{Ethical Implications of AI in Aviation}. Available at: \href{https://doi.org/10.1016/j.dsm.2024.11.001}{DOI Link}.
    \item Author 4. (2024). \textit{Flight Delay Prediction System}. Available at: \href{https://www.researchgate.net/publication/341872407_Flight_Delay_Prediction_System}{ResearchGate}.
    \item Author 5. (2024). \textit{Artificial Intelligence in Aviation Safety: Systematic Review and Biometric Analysis}. Available at: \href{https://www.researchgate.net/publication/385742899_Artificial_Intelligence_in_Aviation_Safety_Systematic_Review_and_Biometric_Analysis}{ResearchGate}.
\end{enumerate}

\end{document}
